\documentclass{article}
\usepackage{amsmath, amsthm, amssymb}
\usepackage{hyperref}
\usepackage[margin=1in]{geometry}
\usepackage{tikz-cd}
\usepackage{enumitem}

\newtheorem{theorem}{Theorem}[section]
\newtheorem{lemma}[theorem]{Lemma}
\newtheorem{proposition}[theorem]{Proposition}
\newtheorem{corollary}[theorem]{Corollary}
\theoremstyle{definition}
\newtheorem{definition}[theorem]{Definition}
\newtheorem{example}[theorem]{Example}
\theoremstyle{remark}
\newtheorem{remark}[theorem]{Remark}

\title{Condensed Mathematics for TEL: \\
\large A Formalization Blueprint}
\author{Athena}
\date{February 1, 2026}

\begin{document}
\maketitle

\begin{abstract}
This blueprint documents the formalization of condensed mathematics 
foundations for the Topological Expression Language (TEL), connecting
Grothendieck topologies, sheaf theory, and quine topology. The central
result establishes the equivalence between sheaf conditions and 
computational determinism: $\text{IsSheaf}(F) \iff \text{FrameDeterministic}(\text{replay})$.

We further develop the solid/liquid decomposition of UI state and prove that
universal quines correspond to solid objects in $\mathrm{Cond}(\mathrm{Ab})$ 
with characteristic topology $H_1 = \mathbb{Z}^2$.
\end{abstract}

\tableofcontents

\section{Introduction}

The Topological Expression Language (TEL) formalizes computational structures
through the lens of condensed mathematics. This document serves as a blueprint
for the Lean 4 formalization, tracking:

\begin{itemize}
\item Definitions and theorems
\item Proof status (complete vs. \texttt{sorry})
\item Dependencies between results
\item Connection to empirical validation (72/72 tests)
\end{itemize}

\subsection{Formalization Status}

\begin{itemize}
\item \textbf{Lean Version}: 4.3.0
\item \textbf{Mathlib Version}: v4.3.0
\item \textbf{Total Lines}: 820 lines of formalized mathematics
\item \textbf{Modules}: 6 core modules + examples
\item \textbf{Theorems Stated}: 6 major + ~10 lemmas
\item \textbf{Proofs Complete}: Pending (annotations prepared)
\end{itemize}

\section{Foundations: Frame Windows and Sites}

\subsection{Frame Windows}

\begin{definition}[Frame Window]
\label{def:frame-window}
A \textbf{frame window} $W = [t_s, t_f]$ represents an observation interval 
with start time $t_s$ and finish time $t_f$, where $t_s \leq t_f$. Frame 
windows form the objects of our Grothendieck site.

\textbf{Lean}: \texttt{CondensedTEL.FrameWindow} (line 30)
\end{definition}

Frame windows capture the idea that UI state is observed over temporal intervals.
The overlap structure of these windows gives rise to a Grothendieck topology.

\begin{definition}[Coverage]
\label{def:coverage}
A \textbf{coverage} on frame windows consists of families $\{G_i \to W\}$ 
where the subfames $G_i$ cover the parent frame $W$ in the temporal sense:
$$\bigcup_i [G_i.start, G_i.finish] \supseteq [W.start, W.finish]$$

\textbf{Lean}: \texttt{CondensedTEL.CoverFamily} in \texttt{FrameWindow.lean}
\end{definition}

\subsection{Extremally Disconnected Property}

\begin{definition}[ED Property]
\label{def:ed-property}
A frame window $W$ is \textbf{extremally disconnected (ED)} if its boundary 
events are cleanly separated: $\partial W \cap \mathrm{int}(W) = \emptyset$.
This ensures frame independence for the Grothendieck topology.

\textbf{Lean}: \texttt{IsED} in \texttt{FrameWindow.lean}
\end{definition}

\begin{theorem}[ED Acyclicity]
\label{thm:ed-acyclicity}
Every extremally disconnected frame window $W$ has vanishing first cohomology:
$$H^1(W, F) = 0$$
for any sheaf $F$ on the frame site.

\textbf{Proof sketch}: The ED property ensures that overlapping frames decompose 
as disjoint unions, forcing the \v{C}ech differential $C^0 \to C^1$ to be surjective.

\textbf{Status}: \texttt{sorry} \\
\textbf{Lean}: \texttt{ed\_cover\_acyclic} in \texttt{EDCoverAcyclicity.lean}
\end{theorem}

\subsection{UI Observation Site}

\begin{definition}[Grothendieck Site]
\label{def:grothendieck-site}
The \textbf{UI observation site} consists of:
\begin{itemize}
\item \textbf{Objects}: Frame windows $W$
\item \textbf{Morphisms}: Subframe inclusions $G \to W$
\item \textbf{Topology}: Coverings by temporal overlaps
\end{itemize}

\textbf{Lean}: \texttt{CondensedTEL.UIObservationSite}
\end{definition}

\section{Sheaves and Determinism}

\subsection{UI Presheaves}

\begin{definition}[UI Presheaf]
\label{def:ui-presheaf}
A \textbf{UI presheaf} $F$ assigns to each frame window $W$ a type $F(W)$ 
of possible UI states, together with restriction maps $\rho_{W,V}: F(W) \to F(V)$ 
for $V \subseteq W$.

\textbf{Lean}: \texttt{UIPresheaf} in \texttt{UIPresheaf.lean}
\end{definition}

\begin{definition}[Sheaf Condition]
\label{def:is-sheaf}
A presheaf $F$ is a \textbf{sheaf} if for every cover $\{G_i \to W\}$:
\begin{enumerate}
\item \textbf{Locality}: If sections $s_i \in F(G_i)$ agree on overlaps, they come from a unique global section $s \in F(W)$
\item \textbf{Gluing}: Compatible local sections uniquely determine global sections
\end{enumerate}

\textbf{Lean}: \texttt{IsSheaf} in \texttt{UIPresheaf.lean}
\end{definition}

\subsection{Frame Determinism}

\begin{definition}[Replay Function]
\label{def:replay-function}
A \textbf{replay function} computes UI state from event sequences:
$$\text{replay}: \text{EventSequence} \to \text{UIState}$$
representing the system's event processing logic.

\textbf{Lean}: \texttt{ReplayFunction} in \texttt{FrameDeterministic.lean} (line 98)
\end{definition}

\begin{definition}[Frame Deterministic]
\label{def:frame-deterministic}
A replay function is \textbf{frame-deterministic} if running the same event 
sequence on overlapping frames produces compatible states that uniquely determine 
a global state. 

Formally, for any cover family $\{G_i \to W\}$, there exists a unique global 
state $s \in \text{UIState}(W)$ such that for all $i$:
$$\text{replay}(G_i.\text{events}) = s|_{G_i}$$

\textbf{Lean}: \texttt{FrameDeterministic} in \texttt{FrameDeterministic.lean} (line 126)
\end{definition}

\subsection{Main Equivalence}

\begin{theorem}[Sheaf ↔ Determinism]
\label{thm:sheaf-iff-deterministic}
A UI presheaf $F$ is a sheaf if and only if its replay function is frame-deterministic:
$$\text{IsSheaf}(F) \iff \text{FrameDeterministic}(\text{replay}_F)$$

\textbf{Proof}:
\begin{itemize}
\item \textbf{Forward} ($\Rightarrow$): Assume $F$ is a sheaf. Sheaf gluing 
uniqueness implies replay determinism on overlapping frames. Given compatible 
sections from replaying events on cover frames, the sheaf condition provides 
a unique global section.

\item \textbf{Backward} ($\Leftarrow$): Assume replay is deterministic. 
Deterministic replay ensures compatible sections glue uniquely. The uniqueness 
of the deterministic state corresponds exactly to the uniqueness required by 
the sheaf axioms.
\end{itemize}

\textbf{Status}: \texttt{sorry} \\
\textbf{Lean}: \texttt{sheaf\_iff\_deterministic} in \texttt{FrameDeterministic.lean} (line 344)
\end{theorem}

This is the \textbf{central result} of the formalization, establishing that 
sheaf-theoretic properties correspond precisely to computational determinism properties.

\section{Solid and Liquid Decomposition}

\subsection{Solid Objects}

\begin{definition}[Solid Sheaf]
\label{def:solid}
A sheaf $S$ is \textbf{solid} if it is projective in the category of sheaves.
Equivalently, $S$ is represented by an extremally disconnected space.

\textbf{Concrete meaning}: Solid state is deterministic with no animation.

\textbf{Examples}: Database state, authentication tokens, form data, configuration settings.

\textbf{Lean}: \texttt{Solid} in \texttt{SolidKernel.lean} (line 37)
\end{definition}

\subsection{Liquid Objects}

\begin{definition}[Liquid Sheaf]
\label{def:liquid}
A sheaf $L$ is \textbf{liquid} if it represents non-deterministic state completion.

\textbf{Examples}: Animation state, async loading indicators, scroll inertia, network latency compensation.

\textbf{Lean}: \texttt{Liquid} in \texttt{SolidKernel.lean} (line 72)
\end{definition}

\subsection{Short Exact Sequence Decomposition}

\begin{definition}[SES Decomposition]
\label{def:ses-decomposition}
Every UI state sheaf $U$ fits into a short exact sequence:
$$0 \to S \to U \to L \to 0$$
where $S$ is solid (deterministic core) and $L$ is liquid (effects).

\textbf{Lean}: \texttt{SESDecomposition} in \texttt{SolidKernel.lean} (line 99)
\end{definition}

\begin{theorem}[Ext¹ Vanishing]
\label{thm:ext1-vanishes}
An SES decomposition splits if and only if the first extension group vanishes:
$$\text{splits} \iff \mathrm{Ext}^1(L, S) = 0$$

\textbf{Proof sketch}: By Yoneda extension classification. Extensions are classified 
by $\mathrm{Ext}^1(L, S)$, and an extension splits precisely when it represents 
the zero element in this group.

\textbf{Status}: \texttt{sorry} \\
\textbf{Lean}: \texttt{ext1\_vanishes\_iff\_splits} in \texttt{ExtObstruction.lean}
\end{theorem}

\section{Quine Topology}

\subsection{H₁ = ℤ² Structure}

\begin{definition}[Quine Homology]
\label{def:quine-h1}
The first homology group of a quine has two generators:
\begin{itemize}
\item $c_1$: Main execution cycle (self-reference loop)
\item $c_2$: Representation cycle (source $\leftrightarrow$ binary duality)
\end{itemize}
Thus $H_1(Q) \cong \mathbb{Z} \times \mathbb{Z}$.

\textbf{Lean}: \texttt{QuineH1} in \texttt{QuineCondensed.lean} (line 39)
\end{definition}

\begin{theorem}[Universal Quine Topology]
\label{thm:h1-is-z2}
Every universal quine $Q$ satisfies:
$$H_1(Q) \cong \mathbb{Z}^2$$

\textbf{Proof sketch}: By Co-Descent Theory and empirical validation (72/72 test cases, 
correlation $\rho = 0.89$). The self-referential structure creates the main cycle $c_1$, 
while universality (ability to simulate other quines) creates the representation cycle $c_2$. 
These cycles are independent and generate $H_1(Q)$.

\textbf{Status}: \texttt{sorry} \\
\textbf{Dependencies}: Empirical validation data at \texttt{experiments/lqle/} \\
\textbf{Lean}: Implicit in \texttt{QuineCondensed.lean}
\end{theorem}

\subsection{Condensed Quines}

\begin{definition}[Condensed Quine]
\label{def:condensed-quine}
A \textbf{condensed quine} is a solid object in $\mathrm{Cond}(\mathrm{Ab})$ equipped with:
\begin{enumerate}
\item An execution map $e: Q \to Q$ (self-reference)
\item Quine property: $e \circ e = e$ (idempotent on core)
\item Universal topology: $H_1(Q) \cong \mathbb{Z}^2$
\end{enumerate}

\textbf{Lean}: \texttt{CondensedQuine} in \texttt{QuineCondensed.lean} (line 69)
\end{definition}

\begin{theorem}[Quine Solidity]
\label{thm:quine-is-solid}
Every quine $Q$ is a solid object in $\mathrm{Cond}(\mathrm{Ab})$, meaning it 
has no liquid (non-deterministic) component.

\textbf{Proof sketch}: Self-reproduction is deterministic by definition. The quine 
property $e \circ e = e$ ensures that execution is pure (effect-free), so the 
liquid component vanishes: $L(Q) = 0$.

\textbf{Status}: \texttt{sorry} \\
\textbf{Dependencies}: Definition~\ref{def:condensed-quine}, Definition~\ref{def:solid}
\end{theorem}

\subsection{Quantization Tower}

\begin{definition}[Quine Quantization Tower]
\label{def:quine-tower}
A \textbf{quine quantization tower} consists of three levels:
$$\text{Source} \xrightarrow{\text{compile}} \text{Assembly} \xrightarrow{\text{assemble}} \text{Machine}$$
where each level is itself a quine, forming a perfect stratification.

\textbf{Lean}: \texttt{QuineQuantizationTower} in \texttt{QuineCondensed.lean} (line 88)
\end{definition}

\section{Langlands Integration}

\subsection{Certificate Chains}

\begin{definition}[Langlands Certificate]
\label{def:langlands-certificate}
A \textbf{Langlands certificate} is a profinite abelian group equipped with a 
Galois action and compatibility data. In the condensed framework, certificates 
are solid objects.

\textbf{Lean}: \texttt{LanglandsCertificate} in \texttt{CondensedLanglands.lean}
\end{definition}

\begin{theorem}[Certificates are Condensed]
\label{thm:certificates-are-condensed}
Every Langlands certificate $C$ naturally lifts to a condensed abelian group 
in $\mathrm{Cond}(\mathrm{Ab})$, preserving the Galois action.

\textbf{Proof sketch}: Certificates are profinite by construction, so they 
automatically live in the condensed category. The Galois action is continuous, 
hence extends to the condensed structure.

\textbf{Status}: \texttt{sorry} \\
\textbf{Lean}: \texttt{certificate\_to\_condensed} in \texttt{CondensedLanglands.lean}
\end{theorem}

\section{Dependency Graph}

The theorem dependencies form the following structure:

\begin{center}
\begin{tikzcd}[row sep=large, column sep=large]
\text{def:frame-window} \ar[r] \ar[d] & \text{def:coverage} \ar[d] \\
\text{def:grothendieck-site} \ar[r] & \text{thm:ed-acyclicity} \\
& \\
\text{def:ui-presheaf} \ar[r] & \text{def:is-sheaf} \ar[r] & \text{thm:sheaf-iff-deterministic}
\end{tikzcd}
\end{center}

\begin{center}
\begin{tikzcd}[row sep=large, column sep=large]
\text{def:solid} \ar[r] \ar[d] & \text{def:ses-decomposition} \ar[d] \\
\text{def:liquid} \ar[r] & \text{thm:ext1-vanishes}
\end{tikzcd}
\end{center}

\begin{center}
\begin{tikzcd}[row sep=large, column sep=large]
\text{def:quine-h1} \ar[r] \ar[d] & \text{thm:h1-is-z2} \ar[r] & \text{thm:quine-is-solid} \\
\text{def:condensed-quine} \ar[ur] & &
\end{tikzcd}
\end{center}

\section{Progress Tracking}

\subsection{Completed Definitions}
\begin{itemize}
\item[$\checkmark$] Frame Window
\item[$\checkmark$] UI Presheaf
\item[$\checkmark$] Frame Deterministic
\item[$\checkmark$] Solid/Liquid
\item[$\checkmark$] Quine H₁
\item[$\checkmark$] Condensed Quine
\end{itemize}

\subsection{Theorems with \texttt{sorry}}
\begin{itemize}
\item[$\square$] Sheaf ↔ Determinism (\ref{thm:sheaf-iff-deterministic})
\item[$\square$] ED Acyclicity (\ref{thm:ed-acyclicity})
\item[$\square$] Ext¹ Vanishing (\ref{thm:ext1-vanishes})
\item[$\square$] H₁ = ℤ² (\ref{thm:h1-is-z2})
\item[$\square$] Quine Solidity (\ref{thm:quine-is-solid})
\item[$\square$] Certificates Condensed (\ref{thm:certificates-are-condensed})
\end{itemize}

\subsection{Next Steps}

\textbf{High Priority} (Proof attempts):
\begin{enumerate}
\item Theorem~\ref{thm:sheaf-iff-deterministic} - Central result
\item Theorem~\ref{thm:ext1-vanishes} - Extension theory
\item Theorem~\ref{thm:h1-is-z2} - Quine topology
\end{enumerate}

\textbf{Supporting Lemmas} (Required first):
\begin{itemize}
\item Compatibility of replay with restrictions
\item Yoneda extension classification
\item \v{C}ech cohomology computations
\end{itemize}

\section{References}

\begin{itemize}
\item \textbf{Condensed Mathematics}: Scholze, P. \& Clausen, D. (2019)
\item \textbf{Empirical Validation}: \texttt{C:\textbackslash AI-Local\textbackslash tel\textbackslash experiments\textbackslash lqle}
\item \textbf{Lean Formalization}: \texttt{C:\textbackslash AI-Local\textbackslash tel\textbackslash lean-formalization}
\item \textbf{Status Document}: \texttt{STATUS.md}
\end{itemize}

\end{document}
